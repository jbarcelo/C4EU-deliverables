The proposed spectrum sensor gathers power metrics at a determined center frequency $f_{0}$ during a fixed \emph{dwelling time} $d_{w}=0.001$~s, moving from $f_{min}$ towards $f_{max}$ at $SS$ frequency intervals. We denote the signal power at interval $i$ as $P_i$ and $T=(f_{max}-f_{min}) / SS$ samples are taken in total. Since $SS$ is typically smaller than the width of a TV channel, several samples are taken for each TV channel.

The procedure described above is conventionally performed using a spectrum analyzer. The resolution bandwidth in the spectrum analyzer is equivalent to our $SS$. In our prototype, there is not a concept equivalent to the spectrum analyzer's video bandwidth, as we do not apply any further filtering on the $P_{i}$ values.

\subsection{Identifying TV White Spaces}

After all the signal power readings (or collection) are contained in a file, a threshold ($\gamma$ in~(\ref{eq:threshold})) is defined.

\begin{equation} \label{eq:threshold}
 \gamma = avg(\min{P_{i}},\max{P_{i}})
\end{equation}

If a received power in the collection ($P_i$, $i=1,2,...,T$) is greater than $\gamma$, then the whole channel at which the sample belongs to is considered \emph{occupied}. On the contrary case, it is considered \emph{free} if $P_{i} \le \gamma$. This measure is used in order to avoid channels where narrow-band transmissions might be present (like wireless microphones).

A graphical representation is found in Fig.~\ref{fig:tvChannels}, where $f_{min}=471.25\  \rm{MHz}$, $f_{max}=863.25\  \rm{MHz}$ and $SS=250\  \rm{kHz}$, resulting in thirty two power samples per TV channel. 