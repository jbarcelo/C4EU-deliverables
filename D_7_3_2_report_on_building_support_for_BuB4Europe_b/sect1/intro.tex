Work Package 7 (WP7), planned for the second and the third and last year of the Commons for Europe project, was based on the following premisses. First, although there are community networks (CNs) and Bottom-up Broadband (BuB) initiatives all around the world and some of them being rather active and successful, the movement remains largely unknown to the general public, and to the public administrations and the policy makers in particular. Second, although these initiatives have many aspects in common such as targets, principals, background, etc., the degree of interaction and collaboration among them is negligible in general. As a result, many of them are tackling with the same obstacles alone and only with their own resources.

The working hypothesis for T7.2, \emph{Enlisting BuB organizations}, and T7.3, \emph{Building Support for BuB4Europe}, was that an international organisation would provide structure and support to the existing BuB initiatives, would stimulate the emergence of new ones, and would contribute to the effectiveness of dissemination, lobbying and influencing the policy makers processes. Nonetheless, after several meetings with practitioners of many initiatives and public administrations, we determined that such organisation was meaningless because, although its creation could be significantly sponsored by C4EU project, its sustainability was not guaranteed. Indeed, those organisations which showed interest, that is to say, BuB initiatives and small local administrations, had not the resources to contribute to it and those which, in our opinion, had the capacity to contribute were not interested in participating. Moreover, we also concluded that the general opinion among the practitioners that contribute in the already existing tools (e.g. maling lists, international meetings, etc.) is that these tools meet their requirements and are adjusted to their capabilities (i.e. light structures with light maintenance costs).

Nevertheless, during our investigations we found that there were tools that could be beneficial for the BuB initiatives were not in place. Given this reality, we decided to reorientate most of the remaining resources that were allocated to the creation and promotion of the formal organisation towards compensating these lacks. Some of the tools have been developed from scratch such as a proposal for the systematisation of the resolution of conflicts, or a methodology for the balance of OPEX and CAPEX among the professionals using the infrastructure held in commons, and others have been taken from guifi.net, such as the translation of the guifi.net license into English, etc. During this year we have also contributed to the maintenance and improvement of those tools that showed to be useful during the first year, that is to say, the BuB mailing list and the bubforeurope.net website.

The remaining of this document is structured as follows. Section \ref{sec:dissemination} accounts for the activities aimed at enlisting organizations and building support for BuB initiatives and section \ref{sec:tools} for the tools for BuB initiatives that either have been developed or improved. Sustainability of the outcome of WP7 and future work are analysed in Section \ref{sec:sust}. Finally results are discussed in Section \ref{sec:results} and the conclusions are presented in Section \ref{sec:conclusions}.
