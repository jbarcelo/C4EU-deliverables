We eventually decided not to create a formal organisation as initially planned because we realised that, firstly, there are examples which show that a minimum amount of resources must be committed to an organisation to make it operate properly, and secondly, it was not possible to gather the required commitment to guarantee these resources once the project finished. The comparison of abandoned organisations such as The Open Spectrum Alliance\footnote{At the beginning being this association was very productive and effective in fulfilling its mission, i.e. lobbing the European Commission in favour of allocating more unlicensed electromagnetic spectrum, but once the individual who was doing the vast majority of the work left (at the time of starting the organisation he was unemployed but then he found a job) the activity suddenly ceased and nothing else was done. At the moment, although the domain is registered, the website (\url{http://openspectrum.eu}) has been down for several months.} with active ones such as the Free Software Foundation Europe (FSFE)\footnote{\url{https://fsfe.org/index.en.html}} evidenced that to ensure the success of the organisations these resources, either economic or human, must be in place. In addition, we observed that the general feeling among the practitioners is that the already existing tools such as the International Summit for Community Wireless Netowrks (IS4CWN)\footnote{http://wirelesssummit.org/} and the Wirless Battlem Mesh (WBM)\footnote{http://battlemesh.org/} gathering events and the FNF and guifi.net websites and mailing lists suffice for the coordination among their organizations, and that these tools are adjusted to their realities. The possibility of not creating the formal organisation was already explained during the second review meeting and was accepted by the Project Officer.

During our meetings to investigate about the conscience of the creation of a formal entity to support BuB initiatives we realised that some of the tools we developed during this reporting period (e.g. the English version of the FONN Compact) were, somehow, expected to have already been put in place by guifi. On the contrary, some others such as the economic compensation or the conflicts resolution systems have been received with great expectation and were totally unforeseen.

The Free Network Foudnation has contributed to the translation of the FONN Compact and has integrated its preamble in their license. Now the efforts are put in the translation of the conflicts resolution system. The ITRF GAIA research group\footnote{Global Access to the Internet for All\url{https://trac.tools.ietf.org/group/irtf/trac/wiki/gaia}} has welcomed the guifi.net tools.

In guifi.net, the economic compensation system has already been implemented in 3 PoPIX and it is expected that the rest will adopt it in the coming months. PoPIX set up from now on will include this system since the begining.

Now that the OF as well as the wireless hybrid nodes is supported, the efforts have been focused in registering the already deployed infrastructure.

