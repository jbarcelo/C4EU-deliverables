
%% bare_jrnl.tex
%% V1.3
%% 2007/01/11
%% by Michael Shell
%% see http://www.michaelshell.org/
%% for current contact information.
%%
%% This is a skeleton file demonstrating the use of IEEEtran.cls
%% (requires IEEEtran.cls version 1.7 or later) with an IEEE journal paper.
%%
%% Support sites:
%% http://www.michaelshell.org/tex/ieeetran/
%% http://www.ctan.org/tex-archive/macros/latex/contrib/IEEEtran/
%% and
%% http://www.ieee.org/



% *** Authors should verify (and, if needed, correct) their LaTeX system  ***
% *** with the testflow diagnostic prior to trusting their LaTeX platform ***
% *** with production work. IEEE's font choices can trigger bugs that do  ***
% *** not appear when using other class files.                            ***
% The testflow support page is at:
% http://www.michaelshell.org/tex/testflow/


%%*************************************************************************
%% Legal Notice:
%% This code is offered as-is without any warranty either expressed or
%% implied; without even the implied warranty of MERCHANTABILITY or
%% FITNESS FOR A PARTICULAR PURPOSE! 
%% User assumes all risk.
%% In no event shall IEEE or any contributor to this code be liable for
%% any damages or losses, including, but not limited to, incidental,
%% consequential, or any other damages, resulting from the use or misuse
%% of any information contained here.
%%
%% All comments are the opinions of their respective authors and are not
%% necessarily endorsed by the IEEE.
%%
%% This work is distributed under the LaTeX Project Public License (LPPL)
%% ( http://www.latex-project.org/ ) version 1.3, and may be freely used,
%% distributed and modified. A copy of the LPPL, version 1.3, is included
%% in the base LaTeX documentation of all distributions of LaTeX released
%% 2003/12/01 or later.
%% Retain all contribution notices and credits.
%% ** Modified files should be clearly indicated as such, including  **
%% ** renaming them and changing author support contact information. **
%%
%% File list of work: IEEEtran.cls, IEEEtran_HOWTO.pdf, bare_adv.tex,
%%                    bare_conf.tex, bare_jrnl.tex, bare_jrnl_compsoc.tex
%%*************************************************************************

% Note that the a4paper option is mainly intended so that authors in
% countries using A4 can easily print to A4 and see how their papers will
% look in print - the typesetting of the document will not typically be
% affected with changes in paper size (but the bottom and side margins will).
% Use the testflow package mentioned above to verify correct handling of
% both paper sizes by the user's LaTeX system.
%
% Also note that the "draftcls" or "draftclsnofoot", not "draft", option
% should be used if it is desired that the figures are to be displayed in
% draft mode.
%
\documentclass[draftclsnofoot,12pt,journal,onecolumn]{IEEEtran}
\renewcommand{\rmdefault}{phv} % Arial
\renewcommand{\sfdefault}{phv} % Arial
\setcounter{page}{3}

%
% If IEEEtran.cls has not been installed into the LaTeX system files,
% manually specify the path to it like:
% \documentclass[journal]{../sty/IEEEtran}





% Some very useful LaTeX packages include:
% (uncomment the ones you want to load)


% *** MISC UTILITY PACKAGES ***
%
\usepackage{ifpdf}
% Heiko Oberdiek's ifpdf.sty is very useful if you need conditional
% compilation based on whether the output is pdf or dvi.
% usage:
% \ifpdf
%   % pdf code
% \else
%   % dvi code
% \fi
% The latest version of ifpdf.sty can be obtained from:
% http://www.ctan.org/tex-archive/macros/latex/contrib/oberdiek/
% Also, note that IEEEtran.cls V1.7 and later provides a builtin
% \ifCLASSINFOpdf conditional that works the same way.
% When switching from latex to pdflatex and vice-versa, the compiler may
% have to be run twice to clear warning/error messages.






% *** CITATION PACKAGES ***
%
\usepackage{cite}
% cite.sty was written by Donald Arseneau
% V1.6 and later of IEEEtran pre-defines the format of the cite.sty package
% \cite{} output to follow that of IEEE. Loading the cite package will
% result in citation numbers being automatically sorted and properly
% "compressed/ranged". e.g., [1], [9], [2], [7], [5], [6] without using
% cite.sty will become [1], [2], [5]--[7], [9] using cite.sty. cite.sty's
% \cite will automatically add leading space, if needed. Use cite.sty's
% noadjust option (cite.sty V3.8 and later) if you want to turn this off.
% cite.sty is already installed on most LaTeX systems. Be sure and use
% version 4.0 (2003-05-27) and later if using hyperref.sty. cite.sty does
% not currently provide for hyperlinked citations.
% The latest version can be obtained at:
% http://www.ctan.org/tex-archive/macros/latex/contrib/cite/
% The documentation is contained in the cite.sty file itself.



%\usepackage{hyperref}


% *** GRAPHICS RELATED PACKAGES ***
%
\ifCLASSINFOpdf
  % \usepackage[pdftex]{graphicx}
  % declare the path(s) where your graphic files are
  % \graphicspath{{../pdf/}{../jpeg/}}
  % and their extensions so you won't have to specify these with
  % every instance of \includegraphics
  % \DeclareGraphicsExtensions{.pdf,.jpeg,.png}
\else
  % or other class option (dvipsone, dvipdf, if not using dvips). graphicx
  % will default to the driver specified in the system graphics.cfg if no
  % driver is specified.
   \usepackage[dvips]{graphicx}
  % declare the path(s) where your graphic files are
   \graphicspath{{./figures/}}
  % and their extensions so you won't have to specify these with
  % every instance of \includegraphics
  % \DeclareGraphicsExtensions{.eps}
\fi
% graphicx was written by David Carlisle and Sebastian Rahtz. It is
% required if you want graphics, photos, etc. graphicx.sty is already
% installed on most LaTeX systems. The latest version and documentation can
% be obtained at: 
% http://www.ctan.org/tex-archive/macros/latex/required/graphics/
% Another good source of documentation is "Using Imported Graphics in
% LaTeX2e" by Keith Reckdahl which can be found as epslatex.ps or
% epslatex.pdf at: http://www.ctan.org/tex-archive/info/
%
% latex, and pdflatex in dvi mode, support graphics in encapsulated
% postscript (.eps) format. pdflatex in pdf mode supports graphics
% in .pdf, .jpeg, .png and .mps (metapost) formats. Users should ensure
% that all non-photo figures use a vector format (.eps, .pdf, .mps) and
% not a bitmapped formats (.jpeg, .png). IEEE frowns on bitmapped formats
% which can result in "jaggedy"/blurry rendering of lines and letters as
% well as large increases in file sizes.
%
% You can find documentation about the pdfTeX application at:
% http://www.tug.org/applications/pdftex





% *** MATH PACKAGES ***
%
\usepackage[cmex10]{amsmath}
\usepackage{amsfonts}
% A popular package from the American Mathematical Society that provides
% many useful and powerful commands for dealing with mathematics. If using
% it, be sure to load this package with the cmex10 option to ensure that
% only type 1 fonts will utilized at all point sizes. Without this option,
% it is possible that some math symbols, particularly those within
% footnotes, will be rendered in bitmap form which will result in a
% document that can not be IEEE Xplore compliant!
%
% Also, note that the amsmath package sets \interdisplaylinepenalty to 10000
% thus preventing page breaks from occurring within multiline equations. Use:
%\interdisplaylinepenalty=2500
% after loading amsmath to restore such page breaks as IEEEtran.cls normally
% does. amsmath.sty is already installed on most LaTeX systems. The latest
% version and documentation can be obtained at:
% http://www.ctan.org/tex-archive/macros/latex/required/amslatex/math/





% *** SPECIALIZED LIST PACKAGES ***
%
%\usepackage{algorithmic}
% algorithmic.sty was written by Peter Williams and Rogerio Brito.
% This package provides an algorithmic environment fo describing algorithms.
% You can use the algorithmic environment in-text or within a figure
% environment to provide for a floating algorithm. Do NOT use the algorithm
% floating environment provided by algorithm.sty (by the same authors) or
% algorithm2e.sty (by Christophe Fiorio) as IEEE does not use dedicated
% algorithm float types and packages that provide these will not provide
% correct IEEE style captions. The latest version and documentation of
% algorithmic.sty can be obtained at:
% http://www.ctan.org/tex-archive/macros/latex/contrib/algorithms/
% There is also a support site at:
% http://algorithms.berlios.de/index.html
% Also of interest may be the (relatively newer and more customizable)
% algorithmicx.sty package by Szasz Janos:
% http://www.ctan.org/tex-archive/macros/latex/contrib/algorithmicx/




% *** ALIGNMENT PACKAGES ***
%
%\usepackage{array}
% Frank Mittelbach's and David Carlisle's array.sty patches and improves
% the standard LaTeX2e array and tabular environments to provide better
% appearance and additional user controls. As the default LaTeX2e table
% generation code is lacking to the point of almost being broken with
% respect to the quality of the end results, all users are strongly
% advised to use an enhanced (at the very least that provided by array.sty)
% set of table tools. array.sty is already installed on most systems. The
% latest version and documentation can be obtained at:
% http://www.ctan.org/tex-archive/macros/latex/required/tools/


%\usepackage{mdwmath}
%\usepackage{mdwtab}
% Also highly recommended is Mark Wooding's extremely powerful MDW tools,
% especially mdwmath.sty and mdwtab.sty which are used to format equations
% and tables, respectively. The MDWtools set is already installed on most
% LaTeX systems. The lastest version and documentation is available at:
% http://www.ctan.org/tex-archive/macros/latex/contrib/mdwtools/


% IEEEtran contains the IEEEeqnarray family of commands that can be used to
% generate multiline equations as well as matrices, tables, etc., of high
% quality.


%\usepackage{eqparbox}
% Also of notable interest is Scott Pakin's eqparbox package for creating
% (automatically sized) equal width boxes - aka "natural width parboxes".
% Available at:
% http://www.ctan.org/tex-archive/macros/latex/contrib/eqparbox/





% *** SUBFIGURE PACKAGES ***
\usepackage[tight,footnotesize]{subfigure}
% subfigure.sty was written by Steven Douglas Cochran. This package makes it
% easy to put subfigures in your figures. e.g., "Figure 1a and 1b". For IEEE
% work, it is a good idea to load it with the tight package option to reduce
% the amount of white space around the subfigures. subfigure.sty is already
% installed on most LaTeX systems. The latest version and documentation can
% be obtained at:
% http://www.ctan.org/tex-archive/obsolete/macros/latex/contrib/subfigure/
% subfigure.sty has been superceeded by subfig.sty.



%\usepackage[caption=false]{caption}
%\usepackage[font=footnotesize,caption=false]{subfig}
% subfig.sty, also written by Steven Douglas Cochran, is the modern
% replacement for subfigure.sty. However, subfig.sty requires and
% automatically loads Axel Sommerfeldt's caption.sty which will override
% IEEEtran.cls handling of captions and this will result in nonIEEE style
% figure/table captions. To prevent this problem, be sure and preload
% caption.sty with its "caption=false" package option. This is will preserve
% IEEEtran.cls handing of captions. Version 1.3 (2005/06/28) and later 
% (recommended due to many improvements over 1.2) of subfig.sty supports
% the caption=false option directly:
%\usepackage[caption=false,font=footnotesize]{subfig}
%
% The latest version and documentation can be obtained at:
% http://www.ctan.org/tex-archive/macros/latex/contrib/subfig/
% The latest version and documentation of caption.sty can be obtained at:
% http://www.ctan.org/tex-archive/macros/latex/contrib/caption/




% *** FLOAT PACKAGES ***
%
%\usepackage{fixltx2e}
% fixltx2e, the successor to the earlier fix2col.sty, was written by
% Frank Mittelbach and David Carlisle. This package corrects a few problems
% in the LaTeX2e kernel, the most notable of which is that in current
% LaTeX2e releases, the ordering of single and double column floats is not
% guaranteed to be preserved. Thus, an unpatched LaTeX2e can allow a
% single column figure to be placed prior to an earlier double column
% figure. The latest version and documentation can be found at:
% http://www.ctan.org/tex-archive/macros/latex/base/



%\usepackage{stfloats}
% stfloats.sty was written by Sigitas Tolusis. This package gives LaTeX2e
% the ability to do double column floats at the bottom of the page as well
% as the top. (e.g., "\begin{figure*}[!b]" is not normally possible in
% LaTeX2e). It also provides a command:
%\fnbelowfloat
% to enable the placement of footnotes below bottom floats (the standard
% LaTeX2e kernel puts them above bottom floats). This is an invasive package
% which rewrites many portions of the LaTeX2e float routines. It may not work
% with other packages that modify the LaTeX2e float routines. The latest
% version and documentation can be obtained at:
% http://www.ctan.org/tex-archive/macros/latex/contrib/sttools/
% Documentation is contained in the stfloats.sty comments as well as in the
% presfull.pdf file. Do not use the stfloats baselinefloat ability as IEEE
% does not allow \baselineskip to stretch. Authors submitting work to the
% IEEE should note that IEEE rarely uses double column equations and
% that authors should try to avoid such use. Do not be tempted to use the
% cuted.sty or midfloat.sty packages (also by Sigitas Tolusis) as IEEE does
% not format its papers in such ways.


%\ifCLASSOPTIONcaptionsoff
%  \usepackage[nomarkers]{endfloat}
% \let\MYoriglatexcaption\caption
% \renewcommand{\caption}[2][\relax]{\MYoriglatexcaption[#2]{#2}}
%\fi
% endfloat.sty was written by James Darrell McCauley and Jeff Goldberg.
% This package may be useful when used in conjunction with IEEEtran.cls'
% captionsoff option. Some IEEE journals/societies require that submissions
% have lists of figures/tables at the end of the paper and that
% figures/tables without any captions are placed on a page by themselves at
% the end of the document. If needed, the draftcls IEEEtran class option or
% \CLASSINPUTbaselinestretch interface can be used to increase the line
% spacing as well. Be sure and use the nomarkers option of endfloat to
% prevent endfloat from "marking" where the figures would have been placed
% in the text. The two hack lines of code above are a slight modification of
% that suggested by in the endfloat docs (section 8.3.1) to ensure that
% the full captions always appear in the list of figures/tables - even if
% the user used the short optional argument of \caption[]{}.
% IEEE papers do not typically make use of \caption[]'s optional argument,
% so this should not be an issue. A similar trick can be used to disable
% captions of packages such as subfig.sty that lack options to turn off
% the subcaptions:
% For subfig.sty:
% \let\MYorigsubfloat\subfloat
% \renewcommand{\subfloat}[2][\relax]{\MYorigsubfloat[]{#2}}
% For subfigure.sty:
% \let\MYorigsubfigure\subfigure
% \renewcommand{\subfigure}[2][\relax]{\MYorigsubfigure[]{#2}}
% However, the above trick will not work if both optional arguments of
% the \subfloat/subfig command are used. Furthermore, there needs to be a
% description of each subfigure *somewhere* and endfloat does not add
% subfigure captions to its list of figures. Thus, the best approach is to
% avoid the use of subfigure captions (many IEEE journals avoid them anyway)
% and instead reference/explain all the subfigures within the main caption.
% The latest version of endfloat.sty and its documentation can obtained at:
% http://www.ctan.org/tex-archive/macros/latex/contrib/endfloat/
%
% The IEEEtran \ifCLASSOPTIONcaptionsoff conditional can also be used
% later in the document, say, to conditionally put the References on a 
% page by themselves.





% *** PDF, URL AND HYPERLINK PACKAGES ***
%
\usepackage{url}
\usepackage[final]{pdfpages}
% url.sty was written by Donald Arseneau. It provides better support for
% handling and breaking URLs. url.sty is already installed on most LaTeX
% systems. The latest version can be obtained at:
% http://www.ctan.org/tex-archive/macros/latex/contrib/misc/
% Read the url.sty source comments for usage information. Basically,
% \url{my_url_here}.


\usepackage{fancyhdr}
\usepackage{lastpage}
\fancyhf{} % sets both header and footer to nothing
\renewcommand{\headrulewidth}{0pt}
\renewcommand{\headheight}{0.4in}
\setlength{\headwidth}{\textwidth}
\fancyhead[L]{
\ifpdf
\includegraphics[height=0.15in]{figures/COM4EU_Logo.pdf}
\else
\includegraphics[height=0.15in]{figures/COM4EU_Logo.eps}
\fi
}
\fancyhead[R]{ % right
C4EU 5.1.3: Report on Selection of Opportunities and Projects -c
}
\pagestyle{fancy}
\cfoot{Page \thepage\ of \pageref{LastPage}}



% *** Do not adjust lengths that control margins, column widths, etc. ***
% *** Do not use packages that alter fonts (such as pslatex).         ***
% There should be no need to do such things with IEEEtran.cls V1.6 and later.
% (Unless specifically asked to do so by the journal or conference you plan
% to submit to, of course. )


% correct bad hyphenation here
\hyphenation{op-tical net-works semi-conduc-tor}


\begin{document}
%
% paper title
% can use linebreaks \\ within to get better formatting as desired
\title{Bottom-up Broadband Pilots in Europe \\ (C4EU 5.1.3: Report on Selection of Opportunities and Projects - c)}
%
%
% author names and IEEE memberships
% note positions of commas and nonbreaking spaces ( ~ ) LaTeX will not break
% a structure at a ~ so this keeps an author's name from being broken across
% two lines.
% use \thanks{} to gain access to the first footnote area
% a separate \thanks must be used for each paragraph as LaTeX2e's \thanks
% was not built to handle multiple paragraphs
%

\author{
	Name~Surname, %\IEEEmembership{Member,~IEEE,}
	Name~Surname, %\IEEEmembership{Member,~IEEE,}
	Name~Surname, %\IEEEmembership{Member,~IEEE,}
	Name~Surname, %\IEEEmembership{Member,~IEEE,}
	Name~Surname, %\IEEEmembership{Member,~IEEE,}
    and~Name~Surname% \IEEEmembership{Life~Fellow,~IEEE}% <-this % stops a space
\thanks{
The authors are with Universitat Pompeu Fabra
}
}


% note the % following the last \IEEEmembership and also \thanks - 
% these prevent an unwanted space from occurring between the last author name
% and the end of the author line. i.e., if you had this:
% 
% \author{....lastname \thanks{...} \thanks{...} }
%                     ^------------^------------^----Do not want these spaces!
%
% a space would be appended to the last name and could cause every name on that
% line to be shifted left slightly. This is one of those "LaTeX things". For
% instance, "\textbf{A} \textbf{B}" will typeset as "A B" not "AB". To get
% "AB" then you have to do: "\textbf{A}\textbf{B}"
% \thanks is no different in this regard, so shield the last } of each \thanks
% that ends a line with a % and do not let a space in before the next \thanks.
% Spaces after \IEEEmembership other than the last one are OK (and needed) as
% you are supposed to have spaces between the names. For what it is worth,
% this is a minor point as most people would not even notice if the said evil
% space somehow managed to creep in.



% The paper headers
%\markboth{\includegraphics[height=0.53in]{figures/COM4EU_Logo.eps} C4EU 5.1.2: Report on Selection of Opportunities and Projects -b}%
\markboth{C4EU 5.1.3: Report on Selection of Opportunities and Projects -c}%
{C4EU 5.1.2: Report on Selection of Opportunities and Projects -b}
% The only time the second header will appear is for the odd numbered pages
% after the title page when using the twoside option.
% 
% *** Note that you probably will NOT want to include the author's ***
% *** name in the headers of peer review papers.                   ***
% You can use \ifCLASSOPTIONpeerreview for conditional compilation here if
% you desire.




% If you want to put a publisher's ID mark on the page you can do it like
% this:
%\IEEEpubid{0000--0000/00\$00.00~\copyright~2007 IEEE}
% Remember, if you use this you must call \IEEEpubidadjcol in the second
% column for its text to clear the IEEEpubid mark.



% use for special paper notices
%\IEEEspecialpapernotice{(Invited Paper)}




% make the title area
\maketitle
%\thispagestyle{fancy}


\begin{abstract}
%\boldmath
This report covers the second call for pilots of the Bottom-up Broadband initiative, the consensus process that led to the definition of the pilots to be executed, and also the teams and pilot charters of the pilots that will be executed.
\end{abstract}
% IEEEtran.cls defaults to using nonbold math in the Abstract.
% This preserves the distinction between vectors and scalars. However,
% if the journal you are submitting to favors bold math in the abstract,
% then you can use LaTeX's standard command \boldmath at the very start
% of the abstract to achieve this. Many IEEE journals frown on math
% in the abstract anyway.

% Note that keywords are not normally used for peerreview papers.
\begin{IEEEkeywords}
 Bottom-up-Broadband (BuB), wifi, fiber, sensor networks, BuB pilots
%Slotted Aloha, game theory, contention control, media access control.
\end{IEEEkeywords}

\clearpage

\tableofcontents

\clearpage

\listoffigures

\listoftables

\clearpage




% For peer review papers, you can put extra information on the cover
% page as needed:
% \ifCLASSOPTIONpeerreview
% \begin{center} \bfseries EDICS Category: 3-BBND \end{center}
% \fi
%
% For peerreview papers, this IEEEtran command inserts a page break and
% creates the second title. It will be ignored for other modes.
\IEEEpeerreviewmaketitle



\section{Introduction}
% The very first letter is a 2 line initial drop letter followed
% by the rest of the first word in caps.
% 
% form to use if the first word consists of a single letter:
% \IEEEPARstart{A}{demo} file is ....
% 
% form to use if you need the single drop letter followed by
% normal text (unknown if ever used by IEEE):
% \IEEEPARstart{A}{}demo file is ....
% 
% Some journals put the first two words in caps:
% \IEEEPARstart{T}{his demo} file is ....
% 
% Here we have the typical use of a "T" for an initial drop letter
% and "HIS" in caps to complete the first word.
\IEEEPARstart{T}{his} report introduces the second round of pilots in the Bottom-up Broadband initiative.
Section \ref{sec:about} explains that it is a collaborative document and how to contribute.
Section \ref{sec:cfp} reproduces the text of the second call for pilots and Section~\ref{sec:received} contains the received proposals.
The combination of different proposals to obtain the pilots to be executed is described in Section~\ref{}.
The team in charge of the execution of the proposals and the pilot charter document for each pilot is introduced in Section~\ref{}.
Finally, Section~\ref{sec:conclusion} concludes the document.


\section{About this document}
\label{sec:about}

This report has been produced using open source tools such as {\LaTeX} \cite{lamport1994ldp} and \emph{git} \cite{chacon2009pg}.
{\LaTeX} is widely used in academia to prepare print-class documents.
It automatically takes care of numbering, cross-referencing, tables of contents, bibliography, etc.
\emph{Git} is a high performance distributed revision control which is used in many open source projects, such as the linux kernel.
Git makes it easy and safe to collaborate as each contributor works on his or her own personal copy.
Good contributions can be easily shared with others, and it is always possible to revert to a previous version.

Our git repository is publicly available in \emph{github}:

https://github.com/jbarcelo/C4EU-deliverables

Anyone who is familiar with {\LaTeX} and \emph{github} can contribute to this document.
The firs step is to make a copy (a \emph{fork} in \emph{github} jargon).
The contributor can work in this copy and make changes to improve the document.
After that, it is necessary to request that these changes are merged into the original copy of the document (a \emph{pull request} in github jargon).

If you see anything that can be improved, feel free to contribute.
This document is alive in the sense that it will keep evolving as long as contributors make changes and improve it.

The system automatically keeps track of all the contributors and their contributions. 
It is possible to see who is contributing more actively and which are the exact changes made by each contributor.
And everything is public on the web.

\section{The second call for pilots}
\label{sec:cfp}

In February 2013 started the dissemination of the second open call for pilots.
The call was first distributed among the \emph{Commons for Europe} partners, then at \emph{BattleMesh} (University of Aalborg) and the TCCC list (The Technical Committee in Computer Communication).
We reproduce here the text of the call:
\begin{quotation}
Dear colleagues,

We are currently studying "Bottom-up Broadband". This is collaborative
grassroots network deployment and maintenance. In this networks, the
users (individuals, institutions, companies or other organizations)
participate in the funding, planning, deployment and maintenance of
the network. If you are participating in one of this initiatives or
are interested in bottom-up-brodband, please contact us.

Open source software has changed the way that software is produced and
maintained. Wikipedia has changed the way encyclopedian information is
compiled and refined. P2P file exchange has changed the way files are
shared and distributed. We believe that collaborative network
deployment can change the way that networks are built, extended and
maintained.

Probably, the most prominent example of collaborative network
deployment is guifi.net. The size of this community network exceeds
20,000 nodes. Last year's efforts have been focusing on community
fiber deployment with around 70 homes and farms connected today.

Besides fiber, we are also interested in ad-hoc mesh networks to cover
events, wireless sensor networks to gather environmental data and
public wifi offering models.

I attach the Bottom-up broadband call for pilots below.

Best regards,
Jaume

******* BOTTOM-UP BROADBAND CALL FOR PILOTS *************

The high expectations created by the European Digital Agenda call for
new models for network deployment. A combination of fiber and wireless
technologies must be part of the solution to achieve the objectives in
2020.


The "Commons for Europe" competitiveness and innovation project
explores "Bottom-up Broadband" (BuB) network deployment initiatives to
analyze the best practices, find replicable success models and offer
guidance to policy makers. In BuB networks, the users play a relevant
role in planning, funding, deploying and maintaining the network. By
users we mean the individuals and organizations, including commercial
companies and public institutions, that benefit from the network.
Network are shared as a "commons" resource by the communities for a
greater benefit (and lower costs) for all the participants. The idea
of BuB is closely linked to that of open access networks, which are
proliferating in northern European countries.


*** BuB Call for pilots 2013 ***


We are looking for BuB intiatives to be considered in the "Commons for
Europe" project. These initiatives will be profusely documented and
used as examples for future BuB network deployments. Each selected
pilot will receive the backing of a BuB fellow for nine months.

Proposal submission deadline: May 15th 2013
Selection decision: June 15th 2013
Start of the pilot phase: July 2013

Visit our web

http://bub4eu.net/

and join our mailing list

https://llistes.guifi.net/sympa/arc/bub

for more information.


A brief description of the first round of executed pilots follows.
(The application form is at the end of the email)
restrictions of the existing infrastructure, allowing mobility and
autonomy is an important field that can benefit all the parts involved
as long as we find the right balance.



Application form:

=========================================================

This pilot project data sheet will help us to keep track of all pilot
initiatives.

Please complete the following fields and submit to jaume.barcelo@upf.edu

(or even better, send it to the bub mailing list if you are registered)

• Title:

• Brief description:

• Goals:

• Estimated start date:

• Estimated end date:

• Priority: (Low/Normal/High)

• Stage: (Prospect/Pre-project/Review/Execution/Evaluation/Finished)

• Type: (Wifi/SuperWifi/Fibre Optics/Hybrid)

• Status: (Not Started/In Progress/On Hold/Completed)

• Progress \%:

• Country:

• Area:

• City:

• Neighbourhood:

• Project type:

• Contacts:

• Risk \%:

(0\% means that the success is guaranteed, 100\% means that it is
impossible to successfully complete this pilot)

• Regulatory issues:

• Potential impact (e.g., number of users, BuB funds raised, cities
involved, etc.):

• Comments:


Thanks for your collaboration :)

End of application form.

=====================================================
\end{quotation}

\section{Received proposals}
\label{sec:received}

\subsection{A Pilot Wireless Network for Remote Sensor Data Collection from Griffin Forest in Central Scotland}

Title: A Pilot Wireless Network for Remote Sensor Data Collection from Griffin Forest in Central
Scotland

Brief description and Goals:

Geoscientists from Edinburgh University have been conducting field experiments since 1996
in the Griffin forest in central Scotland, 4Km southeast of Aberfeldy. The focus of the current
experiment, which is expected to continue for at least 5 years, is on understanding the impact
of aerosol‐based nitrogen fertilization to stimulate forest growth in comparison with the
traditional fertilization method based on solid pellets in significantly higher amounts and
infrequently (typically every 5 years). Understanding the effect of background levels of
nitrogen on forest growth requires monitoring various environmental, micro­climatological
and hydrological factors that together reflect forest growth rate. Consequently, the
experimental site in the Griffin forest has a wide range of sensors deployed over a 1Km2 area.
Currently there is no communications network infrastructure in place at the experimental site for
remote access from Edinburgh as well as for on‐site communication among various
geographically distributed sensors. As a result, geoscientists from Edinburgh connected with
the experiment have to make weekly trips to the site for data collection and maintenance of
site infrastructure (e.g., replacement of batteries on sensors).
The goal of this project is to enable network connectivity to Edinburgh for remote sensor data
collection and site monitoring. This will be achieved by deploying a wireless base station at the
site and a wireless relay that bridges connectivity between the base station and nearest Internet
connection point. Given the remoteness of the site, both these wireless masts have to be
self­powered via renewable energy sources (solar and wind). Long­distance WiFi could be used
to interconnect these wireless masts. Since the vegetation and forest cover dampens wireless
signal propagation, we also intend to experiment with SuperWiFi to understand the benefits it
provides towards achieving required coverage with low power ­­ power conservation is important
when using self­powered wireless relays for reliable connectivity at low cost. The Forestry
Commission Scotland, who have a base near the Griffin site,  have in principle agreed to allow
us to share their Internet connection.

Estimated start date: 1 August 2013

Estimated end date: 30 April 2014

Priority: (Low/Normal/High) High

Stage: (Prospect/Pre­project/Review/Execution/Evaluation/Finished) Pre­project

Type: (Wifi/SuperWifi/Fibre Optics/Hybrid) Wifi and SuperWifi• Status: (Not Started/In Progress/On Hold/Completed) Not Started

Progress %:

Country: UK

Area: Rural

City: Near Aberfeldy, Scotland

Neighbourhood: Griffin Forest

Project type:

Contacts: Dr Mahesh Marina (mmarina@inf.ed.ac.uk)

Risk \%: 0\%. Through the Tegola project \footnote{Tegola Tiered Mesh Network Testbed in Rural Scotland} , we have substantial experience on successfully
using long­distance WiFi for wireless Internet access in rural and remote areas.
(0\% means that the success is guaranteed, 100\% means that it is
impossible to successfully complete this pilot)

Regulatory issues: For experimenting with SuperWiFi, we will get an experimental license from
Ofcom, the UK telecommunications regulating authority.

Potential impact (e.g., number of users, BuB funds raised, cities
involved, etc.):
Having the experiment site connected to the Internet leads to several benefits. Firstly, it will result
in cost savings through reduced travel requirements for sensor data retrieval from the site.
Secondly, it will enable remote monitoring of the experimental system for faults in near real­time
that will in turn lead to data quality improvement and more informed site visits. Thirdly,
networking among sensors within the site will enable development of adaptive sampling systems
which would allow for event driven sampling based on measurements from non‐co‐located
measurement systems. Finally, external communication capability would allow for more rapid
dissemination of results from the experiment to interested communities (e.g., European scale
carbon exchange modelling efforts).

Comments:
We would benefit from ready to use equipment for this pilot approximately valued around £5000,
thanks to the Tegola project which initially was deployed as an experimental wireless testbed but
now has evolved into a self­sustaining rural wireless network owned and run by the local
communities in Northwest Scotland. As already mentioned, this pilot would also benefit from an
Internet connection generously provided by the Forestry Commission Scotland. We would
however need a person to help with the pilot deployment at Griffin site who we hope would be
supported through the BuB initiative.

\section{Pilot Proposal Combination}
\label{sec:Combination}

After receiving all the proposals we started to analyze common trends and possible synergies among them.
The fellows were selected and offered to choose the pilot of their interest.
In what follows, we reproduce the pilot charter of all the pilots that will be executed in the second round of pilots.

\section{Pilot Charter: Mesh Networks of People}
\label{sec:mnp}

Fellow: Sergio Almendros

Mentor: Daniel Mur

Advisor: Jaume Barcelo

\subsection{Pilot purpose or justification}
The purpose of this pilot is to create an android mobile application that allows users to create a mesh network.
The nodes of the network will be the mobile devices running the application, and they will share information among the participating users.

\subsection{Measurable pilot objectives and related success criteria}
\begin{itemize}
\item The android application has to be able to have a list of the people around the smartphone that are using the the application, at least 10 users.
\item If there is a user that is sharing information, other users must be able to see it in less than 1 minute.
\end{itemize}



\section{Conclusion}
\label{sec:conclusion}





% if have a single appendix:
%\appendix[Proof of the Zonklar Equations]
% or
%\appendix  % for no appendix heading
% do not use \section anymore after \appendix, only \section*
% is possibly needed

% use appendices with more than one appendix
% then use \section to start each appendix
% you must declare a \section before using any
% \subsection or using \label (\appendices by itself
% starts a section numbered zero.)
%


%\appendices
%\section{Proof of the First Zonklar Equation}
%Appendix one text goes here.

% you can choose not to have a title for an appendix
% if you want by leaving the argument blank
%\section{}
%Appendix two text goes here.


% use section* for acknowledgement
\section*{Acknowledgment}

This work has been partially funded by the European Commission (grant CIP-ICT PSP-2011-5).
The views expressed in this technical report are solely those of the authors and do not represent the views of the European Commission.


% Can use something like this to put references on a page
% by themselves when using endfloat and the captionsoff option.
%\ifCLASSOPTIONcaptionsoff
%  \newpage
%\fi



% trigger a \newpage just before the given reference
% number - used to balance the columns on the last page
% adjust value as needed - may need to be readjusted if
% the document is modified later
%\IEEEtriggeratref{8}
% The "triggered" command can be changed if desired:
%\IEEEtriggercmd{\enlargethispage{-5in}}

% references section

% can use a bibliography generated by BibTeX as a .bbl file
% BibTeX documentation can be easily obtained at:
% http://www.ctan.org/tex-archive/biblio/bibtex/contrib/doc/
% The IEEEtran BibTeX style support page is at:
% http://www.michaelshell.org/tex/ieeetran/bibtex/
\bibliographystyle{IEEEtran}
% argument is your BibTeX string definitions and bibliography database(s)
\bibliography{IEEEabrv,my_bib}
%
% <OR> manually copy in the resultant .bbl file
% set second argument of \begin to the number of references
% (used to reserve space for the reference number labels box)
%\begin{thebibliography}{1}

%\bibitem{IEEEhowto:kopka}
%H.~Kopka and P.~W. Daly, \emph{A Guide to \LaTeX}, 3rd~ed.\hskip 1em plus
%  0.5em minus 0.4em\relax Harlow, England: Addison-Wesley, 1999.

%\end{thebibliography}

% biography section
% 
% If you have an EPS/PDF photo (graphicx package needed) extra braces are
% needed around the contents of the optional argument to biography to prevent
% the LaTeX parser from getting confused when it sees the complicated
% \includegraphics command within an optional argument. (You could create
% your own custom macro containing the \includegraphics command to make things
% simpler here.)
%\begin{biography}[{\includegraphics[width=1in,height=1.25in,clip,keepaspectratio]{mshell}}]{Michael Shell}
% or if you just want to reserve a space for a photo:

%\begin{IEEEbiography}{Michael Shell}
%Biography text here.
%\end{IEEEbiography}

% if you will not have a photo at all:
%\begin{IEEEbiographynophoto}{John Doe}
%Biography text here.
%\end{IEEEbiographynophoto}

% insert where needed to balance the two columns on the last page with
% biographies
%\newpage

%\begin{IEEEbiographynophoto}{Jane Doe}
%Biography text here.
%\end{IEEEbiographynophoto}

% You can push biographies down or up by placing
% a \vfill before or after them. The appropriate
% use of \vfill depends on what kind of text is
% on the last page and whether or not the columns
% are being equalized.

%\vfill

% Can be used to pull up biographies so that the bottom of the last one
% is flush with the other column.
%\enlargethispage{-5in}

% that's all folks
\end{document}


