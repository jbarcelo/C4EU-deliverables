T7.3 is aimed at enlisting support for the nascent organization of T7.2 and creating a favourable opinion towards it among policymakers that could facilitate its integration in the existing forums. Thus, it has a high dependency on T7.2, specially on the existence of the foreseen organisation, which is expected during the first half of the next reporting period. As a result, most of the efforts of T7.3 are planned for the coming period. The following subsections describe the actions have been carried out during the first reporting.

\subsection{State level}
Interacting with the local governments, from city councils to the Catalan and the Spanish government is part of the standard activities of the guifi.net Foundation. Aside of those contacts that are merely formal administrative and legal proceedings (comunications, notifications, etc.), the rest are aimed at creating awareness of the BuB model and at exploring forms of partnership. In most of these meetings the BuBforEurope initiative has been presented to a greater or lesser degree. A non-exhaustive list of public administrations we presented the initiative follows:

\begin{itemize}
  \setlength{\itemindent}{2em}
  \item \textbf{City Councils}
  \item \textbf{Catalan Government} 
  \item \textbf{Spanish Telecommunications national regulatory authority} 
\end{itemize}


\subsection{European/International level}
Although the vast majority of the efforts on international dissemination and collaboration are planned for the coming year, we have already attended the following international events.

\begin{itemize}
  \item International Summit for Wireless Community Networks 2012, Barcelona (Catalonia), October 2012
  \item FFTH Conference 2013, London (United Kingdom) February 2013
  \item International Summit for Wireless Community Networks 2013, Berlin (Germany), October 2013
\end{itemize}

