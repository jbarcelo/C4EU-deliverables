Despite the fact one of the three pilots is still blocked the progress of the other two together with the results of the other initiatives the experience, as a whole, in our opinion, is a great success. Given the fact that this year there have already been many deployments in parallel, instead of presenting their specific numbers we focus on the most relevant contributions of the BuB model already observed.

\FloatBarrier
\subsection{Model acceptance}
\label{res_sustain}

Despite all the difficulties inherent to new models, the already seen traffic graphs together with the facts that none of the existing users have withdrawn and the future previsions, prove that the model has been widely been accepted.


\FloatBarrier
\subsection{The pilots as a reference}
\label{res_sustain}

The pilots are playing a fundamental role with regards the other projects because they constitute precedents that can be reused in other cases.

\hangindent=2em
\hangafter=1
\textbf{Legal uncertainties clarification}
Although the telecommunications market is liberalised, the facts that it has been a state monopoly for a long time in almost European countries and that it is unusual that the citizens play an active role in it result in a general lack of knowledge regarding the real rights of the citizens and how to exercise them. Learning about these rights and telling to the third parties (specially to the public administrations) has consumed a lot of resources. Having an already running cases together with the support of being a partner of the CommonsforEurope project helps to reduce this overhead in great manner.

\hangindent=2em
\hangafter=1
\textbf{Local agreements}
Agreements to regulate the citizens-private-public collaboration are part of the tools that have had to be developed. The current ones are used as guide for the new ones.

\hangindent=2em
\hangafter=1
\textbf{Best practices}
The day-to-day work is the best way to gain expertise. Best practices regarding technical, financial, etc. aspects are shared among the pilots and the community.


\FloatBarrier
\subsection{Sustainability and Scalability}
\label{res_sustain}

The results achieved, specially the fact that new initiatives are being developed, show that the model is sustainable from the economic point of view and that it is feasible from the technical and the social points of view.


\FloatBarrier
\subsection{Local economic promotion}
\label{res_local_econ}

In addition to the impact in terms of local economic promotion any deployment of an optical fibre network accessible by the inhabitants (essentially in terms of the cost, because, unfortunately, few people is concerned about other aspects like data retention, surveillance, etc.) has, specially if it is the first one available, the BuB model the following traits:

\hangindent=2em
\hangafter=1
\textbf{Knowledge transfer}
All knowledge and information is accessible to the public. Thus everybody has the same opportunities to start a business. Professional secrecy has a extremely bad reputation.

\hangindent=2em
\hangafter=1
\textbf{Job and small business creation}
Many SMEs have been created so far around guifi.net Two types of job positions are the most common so far: Internet service access (over the CN) and physical installations. Frequently SMEs combine both of them and, additionally, take the participation in the project as an opportunity to better position themselves in the market.

\hangindent=2em
\hangafter=1
\textbf{SMEs cooperation}
The guifi.net Foundation strongly promotes the cooperation among SMEs in terms of knowledge sharing, resources sharing, etc. The activity in the concentration PoP is clear example of this collaboration. There ISPs operating in guifi.net (they are all SME) gather together to deal special offers with the providers (carrier, collocation, etc.) and share costs.


\FloatBarrier
\subsection{Digital divide reduction}
\label{res_digital_div}

The presence of a decent internet access has many other benefits for the population aside from the direct impact on the local economy. Nonetheless these benefits are restricted to those who can afford the connection cost. The BuB model is a significant contribution towards the digital divide reduction because BuB is a cost oriented model (fairness in prices is a compromise taken by the ISPs) and also a Do-It-Yourself model (so everybody has the right to make his own deployments).


\FloatBarrier
\subsection{Social Implication}
\label{res_social}

BuB model fosters social implication of individuals in great manner. Things do not happen if people do not want them to happen. After almost ten years of activity people have (re-)learn how to cooperate to achieve a common objective, how to interact with the public administrations, how and why to take care of the common good, etc.

We are convinced that this way of doing things can be exported to many other fields like the electrical power or the health care systems.

