The main conclusion of the results of the Optic Fibre pilots confirm the initial working hypothesis, that is to say, that the fundamentals of the crowdsourcing model successfully put in place during the last decade to boost a telecommunications infrastructure held in commons based on WiFi technologies can be used to deploy Optic Fibre infrastructure in the same commons model.

The execution of these pioneering pilots has significantly contributed in several aspects. Methodologies to make projects involving higher budgets, higher technological complexity and with higher dependencies on local authorities have been developed and put in practice. The software tools as well as the database have had to be engineered to accommodate the new technology requirements. A system to balance the usage and contributions to the infrastructure held in commons by the professionals has been put in place, etc.

During the second reporting period four new operational PoPs have been added to the five already existing ones. Thus, there are already eight on-going deployments in paral·lel and others are expected to consolidate during the next year.

Sustainability of the model is proved by the number of new OF initiatives emerged during the execution of the pilots as well as for the growth of the pilots itself.


As far as end-users accounting metric is concerned, in order to preserve the inherent autonomy of the BuB model and to better adapt to the current number of PoPs and its expected future evolution, in this report we introduced the PoPs traffic at the expenses of the precise number of end-users connected. By the end of this reporting period the aggregated OF transit (internet carrier + CATNIX peers) is above 500Mb/s (95\% percentile). The yearly traffic graphs presented show a clear sustained rising trend. 

In addition to the PoPs and end-user results, the contributions made in local economic promotion, social implication, digital divide reduction, etc. must also be highlighted.

Thus, our overall assessment of T5.4 is extremely positive. We are convinced that through it we have made rather significant contributions to the model development and implementation.
